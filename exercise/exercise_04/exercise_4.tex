\documentclass[12pt,a4paper]{article}
\usepackage{lmodern}

\usepackage{scalerel}
\usepackage{bbm}
\usepackage{enumitem}
\usepackage{placeins}
\usepackage{amssymb,amsmath}
\usepackage{ifxetex,ifluatex}
\usepackage{fixltx2e} % provides \textsubscript
\ifnum 0\ifxetex 1\fi\ifluatex 1\fi=0 % if pdftex
  \usepackage[T1]{fontenc}
  \usepackage[utf8]{inputenc}
\else % if luatex or xelatex
  \ifxetex
    \usepackage{mathspec}
    \usepackage{xltxtra,xunicode}
  \else
    \usepackage{fontspec}
  \fi
  \defaultfontfeatures{Mapping=tex-text,Scale=MatchLowercase}
  \newcommand{\euro}{€}
\fi
% use upquote if available, for straight quotes in verbatim environments
\IfFileExists{upquote.sty}{\usepackage{upquote}}{}
% use microtype if available
\IfFileExists{microtype.sty}{%
\usepackage{microtype}
\UseMicrotypeSet[protrusion]{basicmath} % disable protrusion for tt fonts
}{}
\usepackage[lmargin = 2cm, rmargin = 2.5cm, tmargin = 2cm, bmargin =
2.5cm]{geometry}


% Figure Placement:
\usepackage{float}
\let\origfigure\figure
\let\endorigfigure\endfigure
\renewenvironment{figure}[1][2] {
    \expandafter\origfigure\expandafter[H]
} {
    \endorigfigure
}

%%%% Jens %%%%
\usepackage{titlesec}
\DeclareMathOperator*{\argmax}{arg\,max}
\DeclareMathOperator*{\argmin}{arg\,min}
\renewcommand{\vec}{\operatorname{vec}}
\newcommand{\tr}{\operatorname{tr}}
\newcommand{\Var}{\operatorname{Var}} % Variance
\newcommand{\MSE}{\operatorname{MSE}} % Variance
\newcommand{\VAR}{\operatorname{VAR}} % Vector autoregression
\newcommand{\Lag}{\operatorname{L}} % Lag operator
\newcommand{\Cov}{\operatorname{Cov}}
\newcommand{\diag}{\operatorname{diag}}
\newcommand{\adj}{\operatorname{adj}}
\newcommand{\loglik}{\operatorname{ll}}


\usepackage{booktabs}
\usepackage{centernot}
\usepackage{mathtools}

\allowdisplaybreaks

\titleformat{\section}
{\normalfont\large\bfseries}{\thesection}{1em}{}

\newcommand{\tmpsection}[1]{}
\let\tmpsection=\section
\renewcommand{\section}[1]{\tmpsection{\underline{#1}} }

\providecommand{\tightlist}{%
  \setlength{\itemsep}{0pt}\setlength{\parskip}{0pt}}

%% citation setup
\usepackage{csquotes}

\usepackage{graphicx}
\makeatletter
\def\maxwidth{\ifdim\Gin@nat@width>\linewidth\linewidth\else\Gin@nat@width\fi}
\def\maxheight{\ifdim\Gin@nat@height>\textheight\textheight\else\Gin@nat@height\fi}
\makeatother
% Scale images if necessary, so that they will not overflow the page
% margins by default, and it is still possible to overwrite the defaults
% using explicit options in \includegraphics[width, height, ...]{}
\setkeys{Gin}{width=\maxwidth,height=\maxheight,keepaspectratio}
\ifxetex
  \usepackage[setpagesize=false, % page size defined by xetex
              unicode=false, % unicode breaks when used with xetex
              xetex]{hyperref}
\else
  \usepackage[unicode=true, linktocpage = TRUE]{hyperref}
\fi
\hypersetup{breaklinks=true,
            bookmarks=true,
            pdfauthor={Prof.~Christoph Hanck},
            pdftitle={Deskriptive Statistik},
            colorlinks=true,
            citecolor=black,
            urlcolor=black,
            linkcolor=black,
            pdfborder={0 0 0}}
\urlstyle{same}  % don't use monospace font for urls
\setlength{\parindent}{0pt}
\setlength{\parskip}{6pt plus 2pt minus 1pt}
\setlength{\emergencystretch}{3em}  % prevent overfull lines
\setcounter{secnumdepth}{5}

%%% Use protect on footnotes to avoid problems with footnotes in titles
\let\rmarkdownfootnote\footnote%
\def\footnote{\protect\rmarkdownfootnote}

%%% Change title format to be more compact
\usepackage{titling}

% Create subtitle command for use in maketitle
\newcommand{\subtitle}[1]{
  \posttitle{
    \begin{center}\large#1\end{center}
    }
}

\setlength{\droptitle}{-2em}
  \title{Deskriptive Statistik}
  \pretitle{\vspace{\droptitle}\centering\huge}
  \posttitle{\par}
\subtitle{Übung 4}
  \author{Prof.~Christoph Hanck}
  \preauthor{\centering\large\emph}
  \postauthor{\par}
  \date{}
  \predate{}\postdate{}


%% linespread settings

\usepackage{setspace}

\onehalfspacing


% Language Setup

\usepackage{ifthen}
\usepackage{iflang}
\usepackage[super]{nth}
\usepackage[ngerman, english]{babel}

%Acronyms
\usepackage[printonlyused, withpage, nohyperlinks]{acronym}
\usepackage{changepage}

% Multicols for the Title page
\usepackage{multicol}


% foot


\begin{document}

\selectlanguage{english}

%%%%%%%%%%%%%% Jens %%%%%
\numberwithin{equation}{section}




\restoregeometry


%%% Header 

\begin{minipage}{0.6\textwidth}
Universität Duisburg-Essen\\
Lehrstuhl Ökonometrie\\
Prof.~Christoph Hanck \\
M.Sc. Jens Klenke \\
\end{minipage}

%\begin{minipage}{0.3\textwidth}
	\begin{flushright}
	\vspace{-3.55cm}
	\includegraphics*[width=5cm]{../Includes/duelogo_de.png}\\
	\vspace{.125cm}
	Wintersemester 2022/2023
	\end{flushright}
%\end{minipage}
%\vspace{.125cm}


\begin{center}
	\vspace{.25cm}
	\textbf{\Large{Deskriptive Statistik}}\\
	\vspace{.25cm}
	\textbf{\large{Übung 4}}\\
	\vspace{.125cm}
\end{center}




% body from markdown

\hypertarget{lageparameter-i}{%
\section{Lageparameter I}\label{lageparameter-i}}

Aus der Vorlesung ist Ihnen schon der Datensatz \texttt{iris} bekannt,
welcher Informationen über die Abmessungen von Kelchblättern beinhaltet.
Die Variable \texttt{Sepal.Width} beschreibt die Breite von insgesamt
\(150\) Blättern.

Berechnen Sie für die Breite:

\begin{enumerate}[label=(\alph*)]
  \item den Median. 
  \item den Modus. 
  \item das arithmetische Mittel. 
\end{enumerate}

\emph{Hinweis: Der Datensatz steht Ihnen in \texttt{R} automatisch zur
Verfügung. Um auf eine Variable in einen Datensatz zugreifen zu können,
müssen Sie das \texttt{\$}-Symbol benutzen.}

\hypertarget{lageparameter-ii}{%
\section{Lageparameter II}\label{lageparameter-ii}}

\(134\) Bereitschaften des Malteser Hilfsdienstes werden über die Anzahl
ihrer jährlichen Einsätze befragt. Nachdem \(130\) Bereitschaften ihre
Angaben gemacht haben, ergibt sich:

\FloatBarrier
\begin{table}[htp]
  \begin{tabular}{lr}
    Modus                 & $x_M = 2500$ \\
    Median                & $x_{Med} = 2900$ \\
    arithmetisches Mittel & $\bar{x} = 3200$
  \end{tabular}
\end{table}
\FloatBarrier

Schließlich machen die letzten vier Bereitschaften ihre Angaben. Sie
fuhren \(1000\), \(1800\), \(5000\) bzw. \(6072\) Einsätze.

Bestimmen Sie unter Berücksichtigung aller eingegangenen Daten:

\begin{enumerate}[label=(\alph*)]
  \item den Median. 
  \item den Modus. 
  \item das arithmetische Mittel. 
\end{enumerate}

\hypertarget{minimierungseigenschaft-des-arithmetischen-mittels}{%
\section{Minimierungseigenschaft des arithmetischen
Mittels}\label{minimierungseigenschaft-des-arithmetischen-mittels}}

Zeigen Sie die Minimierungseigenschaf:

\[\sum_{j = 1}^{n} (x_j - \bar{x})^{2} \leq \sum_{j = 1}^{n} (x_j - a)^{2} \qquad \text{für}\; a \in \mathbb{R}\]

\hypertarget{lageparameter-iii}{%
\section{Lageparameter III}\label{lageparameter-iii}}

Der monatliche Gasverbrauch (in Kubikmetern) eines Einfamilienhauses im
Jahr 2021 war wie folgt:

\FloatBarrier
\begin{table}[htp]
  \centering
   \resizebox{\textwidth}{!}{ 
  \begin{tabular}{@{}lcccccccccccc@{}}
    Monat & Jan & Feb & Mär & Apr & Mai & Jun & Jul & Aug & Sep & Okt & Nov & Dez \\ \midrule
    $x_i$ & $386,4$  & $312,0$ & $300,0$ & $194.4$  & $84,0$ &  $52,8$ & $40,8$ & $38,4$ & $124,8$ & $201,6$ & $292,8$ & $372,0$     \\
  \end{tabular}
   }
\end{table}
\FloatBarrier

\begin{enumerate}[label=(\alph*)]
  \item Berechnen Sie den durchschnittlichen Gasverbrauch pro Monat.
  \item Auf der jährlichen Abrechnung des Gasversorgers wird der Verbrauch in der Regel in Kilowattstunden (kWh) angegeben. Dieser Wert ergibt sich aus der Formel
  
  $$kWh = m^3 \times Brennwert \times Zustandszahl$$
  welche die Menge an tatsächlich entnommener thermischer Energie angibt. Angenommen der Brennwert betrage $9,3$ und die Zustandszahl $0,9103$. Wieviel thermische Energie wurde im Durchschnitt im Beispielhaushalt pro Monat verbraucht?
\end{enumerate}

\hypertarget{axiome}{%
\section{Axiome}\label{axiome}}

Erläutern Sie kurz die \(4\) wichtigsten Axiome der Lageparameter und
zeigen Sie, dass jene von Modus \(x_M\), Median \(x_{Med}\) und
arithmetischem Mittel \(\bar{x}\) erfüllt werden.

\end{document}
