\documentclass[12pt,a4paper]{article}
\usepackage{lmodern}

\usepackage{scalerel}
\usepackage{bbm}
\usepackage{enumitem}
\usepackage{placeins}
\usepackage{amssymb,amsmath}
\usepackage{ifxetex,ifluatex}
\usepackage{fixltx2e} % provides \textsubscript
\ifnum 0\ifxetex 1\fi\ifluatex 1\fi=0 % if pdftex
  \usepackage[T1]{fontenc}
  \usepackage[utf8]{inputenc}
\else % if luatex or xelatex
  \ifxetex
    \usepackage{mathspec}
    \usepackage{xltxtra,xunicode}
  \else
    \usepackage{fontspec}
  \fi
  \defaultfontfeatures{Mapping=tex-text,Scale=MatchLowercase}
  \newcommand{\euro}{€}
\fi
% use upquote if available, for straight quotes in verbatim environments
\IfFileExists{upquote.sty}{\usepackage{upquote}}{}
% use microtype if available
\IfFileExists{microtype.sty}{%
\usepackage{microtype}
\UseMicrotypeSet[protrusion]{basicmath} % disable protrusion for tt fonts
}{}
\usepackage[lmargin = 2cm, rmargin = 2.5cm, tmargin = 2cm, bmargin =
2.5cm]{geometry}


% Figure Placement:
\usepackage{float}
\let\origfigure\figure
\let\endorigfigure\endfigure
\renewenvironment{figure}[1][2] {
    \expandafter\origfigure\expandafter[H]
} {
    \endorigfigure
}

%%%% Jens %%%%
\usepackage{titlesec}
\DeclareMathOperator*{\argmax}{arg\,max}
\DeclareMathOperator*{\argmin}{arg\,min}
\renewcommand{\vec}{\operatorname{vec}}
\newcommand{\tr}{\operatorname{tr}}
\newcommand{\Var}{\operatorname{Var}} % Variance
\newcommand{\MSE}{\operatorname{MSE}} % Variance
\newcommand{\VAR}{\operatorname{VAR}} % Vector autoregression
\newcommand{\Lag}{\operatorname{L}} % Lag operator
\newcommand{\Cov}{\operatorname{Cov}}
\newcommand{\diag}{\operatorname{diag}}
\newcommand{\adj}{\operatorname{adj}}
\newcommand{\loglik}{\operatorname{ll}}


\usepackage{booktabs}
\usepackage{centernot}
\usepackage{mathtools}

\allowdisplaybreaks

\titleformat{\section}
{\normalfont\large\bfseries}{\thesection}{1em}{}

\newcommand{\tmpsection}[1]{}
\let\tmpsection=\section
\renewcommand{\section}[1]{\tmpsection{\underline{#1}} }

\providecommand{\tightlist}{%
  \setlength{\itemsep}{0pt}\setlength{\parskip}{0pt}}

%% citation setup
\usepackage{csquotes}

\usepackage{graphicx}
\makeatletter
\def\maxwidth{\ifdim\Gin@nat@width>\linewidth\linewidth\else\Gin@nat@width\fi}
\def\maxheight{\ifdim\Gin@nat@height>\textheight\textheight\else\Gin@nat@height\fi}
\makeatother
% Scale images if necessary, so that they will not overflow the page
% margins by default, and it is still possible to overwrite the defaults
% using explicit options in \includegraphics[width, height, ...]{}
\setkeys{Gin}{width=\maxwidth,height=\maxheight,keepaspectratio}
\ifxetex
  \usepackage[setpagesize=false, % page size defined by xetex
              unicode=false, % unicode breaks when used with xetex
              xetex]{hyperref}
\else
  \usepackage[unicode=true, linktocpage = TRUE]{hyperref}
\fi
\hypersetup{breaklinks=true,
            bookmarks=true,
            pdfauthor={Prof.~Christoph Hanck},
            pdftitle={Deskriptive Statistik},
            colorlinks=true,
            citecolor=black,
            urlcolor=black,
            linkcolor=black,
            pdfborder={0 0 0}}
\urlstyle{same}  % don't use monospace font for urls
\setlength{\parindent}{0pt}
\setlength{\parskip}{6pt plus 2pt minus 1pt}
\setlength{\emergencystretch}{3em}  % prevent overfull lines
\setcounter{secnumdepth}{5}

%%% Use protect on footnotes to avoid problems with footnotes in titles
\let\rmarkdownfootnote\footnote%
\def\footnote{\protect\rmarkdownfootnote}

%%% Change title format to be more compact
\usepackage{titling}

% Create subtitle command for use in maketitle
\newcommand{\subtitle}[1]{
  \posttitle{
    \begin{center}\large#1\end{center}
    }
}

\setlength{\droptitle}{-2em}
  \title{Deskriptive Statistik}
  \pretitle{\vspace{\droptitle}\centering\huge}
  \posttitle{\par}
\subtitle{Übung 5}
  \author{Prof.~Christoph Hanck}
  \preauthor{\centering\large\emph}
  \postauthor{\par}
  \date{}
  \predate{}\postdate{}


%% linespread settings

\usepackage{setspace}

\onehalfspacing


% Language Setup

\usepackage{ifthen}
\usepackage{iflang}
\usepackage[super]{nth}
\usepackage[ngerman, english]{babel}

%Acronyms
\usepackage[printonlyused, withpage, nohyperlinks]{acronym}
\usepackage{changepage}

% Multicols for the Title page
\usepackage{multicol}


% foot


\begin{document}

\selectlanguage{english}

%%%%%%%%%%%%%% Jens %%%%%
\numberwithin{equation}{section}




\restoregeometry


%%% Header 

\begin{minipage}{0.6\textwidth}
Universität Duisburg-Essen\\
Lehrstuhl Ökonometrie\\
Prof.~Christoph Hanck \\
M.Sc. Jens Klenke \\
\end{minipage}

%\begin{minipage}{0.3\textwidth}
	\begin{flushright}
	\vspace{-3.55cm}
	\includegraphics*[width=5cm]{../Includes/duelogo_de.png}\\
	\vspace{.125cm}
	Wintersemester 2022/2023
	\end{flushright}
%\end{minipage}
%\vspace{.125cm}


\begin{center}
	\vspace{.25cm}
	\textbf{\Large{Deskriptive Statistik}}\\
	\vspace{.25cm}
	\textbf{\large{Übung 5}}\\
	\vspace{.125cm}
\end{center}




% body from markdown

\hypertarget{lageparameter-i}{%
\section{Lageparameter I}\label{lageparameter-i}}

\begin{center}
  \begin{tabular}{lcc}
    $j$ & Jahr & $y_j$ \\
    \toprule
    $1$ & $2016$ & $1524,8$ \\
    $2$ & $2017$ & $1601,3$ \\
    $3$ & $2018$ & $1667,9$ \\
    $4$ & $2019$ & $1989,5$ \\
  \end{tabular}
\end{center}

Berechnen Sie die durchschnittliche Wachstumsrate der Staatsausgaben.

\hypertarget{lageparameter-ii}{%
\section{Lageparameter II}\label{lageparameter-ii}}

Eine Person fährt mit ihrem Auto von A nach B und zurück. Die
Streckenlänge beträgt \(100\) km. Auf der Hinfahrt konnte lediglich eine
Durchschnittsgeschwindigkeit von \(50\) km/h erreicht werden, wohingegen
die Rückfahrt mit einer durchschnittlichen Geschwindigkeit von \(100\)
km/h zurückgelegt werden konnte.

Wie hoch war die Durchschnittsgeschwindigkeit beider Fahrten zusammen.

\hypertarget{streuungsmauxdfe}{%
\section{Streuungsmaße}\label{streuungsmauxdfe}}

\begin{center}
  \begin{tabular}{cll}
    $x_i$& $h_i$&  $H(x)$\\
  \toprule
    1&$\frac{5}{40}=0,125$&0,125\\[2mm]
    2&$\frac{7}{40}=0,175$&0,3\\[2mm]
    3&$\frac{18}{40}=0,45$&0,75\\[2mm]
    4&$\frac{8}{40}=0,2$&0,95\\[2mm]
    5&$\frac{2}{40}=0,05$&1\\
  \end{tabular}
\end{center}

\newpage

Berechnen Sie für diesen Datensatz:

\begin{enumerate}[label=(\alph*)]
  \item Spannweite 
  \item Quartilsabstand 
  \item Varianz
\end{enumerate}

\hypertarget{zusuxe4tzliche-beobachtung}{%
\section{Zusätzliche Beobachtung}\label{zusuxe4tzliche-beobachtung}}

Für ein metrisches Merkmal \(X\) liegen 10 Beobachtungen mit einem
arithmetischen Mittel \(\bar{X} = 6,9\) und einer Varianz \(s^2 = 14,2\)
vor.

Welches arithmetische Mittel und welche Standardabweichung ergeben sich
bei Hinzunahme der weiteren Ausprägung \(x_{11} = 8\)?

\hypertarget{streuungsmauxdfe-1}{%
\section{Streuungsmaße}\label{streuungsmauxdfe-1}}

Ein Triathlet möchte sich aus Zeitgründen auf die Sportart beschränken,
in der er die gleichmäßigste Leistung erbrachte (Angaben in Minuten):

\begin{center}
 \begin{tabular}{lcccccc} 
    Training   & 1     &    2  &   3   &    4  &    5  &   6   \\
    \toprule
    Schwimmen  & 20,6  &  22,0 &  19,8 &  20,1 &  21,5 &  20,5 \\ 
    Radfahren  & 45,3  &  44,7 &  40,5 &  49,0 &  46,2 &  47,2 \\ 
    Laufen     & 68,1  &  66,2 &  69,3 &  74,4 &  61,2 &  67,3 \\ 
\end{tabular}
\end{center}

Für welche Sportart wird sich der Athlet Ihrer Meinung nach entscheiden?
Mit welcher Maßzahl würden Sie den Vergleich durchführen?

\end{document}
