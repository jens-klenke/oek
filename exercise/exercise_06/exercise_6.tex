\documentclass[12pt,a4paper]{article}
\usepackage{lmodern}

\usepackage{scalerel}
\usepackage{bbm}
\usepackage{enumitem}
\usepackage{placeins}
\usepackage{amssymb,amsmath}
\usepackage{ifxetex,ifluatex}
\usepackage{fixltx2e} % provides \textsubscript
\ifnum 0\ifxetex 1\fi\ifluatex 1\fi=0 % if pdftex
  \usepackage[T1]{fontenc}
  \usepackage[utf8]{inputenc}
\else % if luatex or xelatex
  \ifxetex
    \usepackage{mathspec}
    \usepackage{xltxtra,xunicode}
  \else
    \usepackage{fontspec}
  \fi
  \defaultfontfeatures{Mapping=tex-text,Scale=MatchLowercase}
  \newcommand{\euro}{€}
\fi
% use upquote if available, for straight quotes in verbatim environments
\IfFileExists{upquote.sty}{\usepackage{upquote}}{}
% use microtype if available
\IfFileExists{microtype.sty}{%
\usepackage{microtype}
\UseMicrotypeSet[protrusion]{basicmath} % disable protrusion for tt fonts
}{}
\usepackage[lmargin = 2cm, rmargin = 2.5cm, tmargin = 2cm, bmargin =
2.5cm]{geometry}


% Figure Placement:
\usepackage{float}
\let\origfigure\figure
\let\endorigfigure\endfigure
\renewenvironment{figure}[1][2] {
    \expandafter\origfigure\expandafter[H]
} {
    \endorigfigure
}

%%%% Jens %%%%
\usepackage{titlesec}
\DeclareMathOperator*{\argmax}{arg\,max}
\DeclareMathOperator*{\argmin}{arg\,min}
\renewcommand{\vec}{\operatorname{vec}}
\newcommand{\tr}{\operatorname{tr}}
\newcommand{\Var}{\operatorname{Var}} % Variance
\newcommand{\MSE}{\operatorname{MSE}} % Variance
\newcommand{\VAR}{\operatorname{VAR}} % Vector autoregression
\newcommand{\Lag}{\operatorname{L}} % Lag operator
\newcommand{\Cov}{\operatorname{Cov}}
\newcommand{\diag}{\operatorname{diag}}
\newcommand{\adj}{\operatorname{adj}}
\newcommand{\loglik}{\operatorname{ll}}


\usepackage{booktabs}
\usepackage{centernot}
\usepackage{mathtools}

\allowdisplaybreaks

\titleformat{\section}
{\normalfont\large\bfseries}{\thesection}{1em}{}

\newcommand{\tmpsection}[1]{}
\let\tmpsection=\section
\renewcommand{\section}[1]{\tmpsection{\underline{#1}} }

\providecommand{\tightlist}{%
  \setlength{\itemsep}{0pt}\setlength{\parskip}{0pt}}

%% citation setup
\usepackage{csquotes}

\usepackage{graphicx}
\makeatletter
\def\maxwidth{\ifdim\Gin@nat@width>\linewidth\linewidth\else\Gin@nat@width\fi}
\def\maxheight{\ifdim\Gin@nat@height>\textheight\textheight\else\Gin@nat@height\fi}
\makeatother
% Scale images if necessary, so that they will not overflow the page
% margins by default, and it is still possible to overwrite the defaults
% using explicit options in \includegraphics[width, height, ...]{}
\setkeys{Gin}{width=\maxwidth,height=\maxheight,keepaspectratio}
\ifxetex
  \usepackage[setpagesize=false, % page size defined by xetex
              unicode=false, % unicode breaks when used with xetex
              xetex]{hyperref}
\else
  \usepackage[unicode=true, linktocpage = TRUE]{hyperref}
\fi
\hypersetup{breaklinks=true,
            bookmarks=true,
            pdfauthor={Prof.~Christoph Hanck},
            pdftitle={Deskriptive Statistik},
            colorlinks=true,
            citecolor=black,
            urlcolor=black,
            linkcolor=black,
            pdfborder={0 0 0}}
\urlstyle{same}  % don't use monospace font for urls
\setlength{\parindent}{0pt}
\setlength{\parskip}{6pt plus 2pt minus 1pt}
\setlength{\emergencystretch}{3em}  % prevent overfull lines
\setcounter{secnumdepth}{5}

%%% Use protect on footnotes to avoid problems with footnotes in titles
\let\rmarkdownfootnote\footnote%
\def\footnote{\protect\rmarkdownfootnote}

%%% Change title format to be more compact
\usepackage{titling}

% Create subtitle command for use in maketitle
\newcommand{\subtitle}[1]{
  \posttitle{
    \begin{center}\large#1\end{center}
    }
}

\setlength{\droptitle}{-2em}
  \title{Deskriptive Statistik}
  \pretitle{\vspace{\droptitle}\centering\huge}
  \posttitle{\par}
\subtitle{Übung 6}
  \author{Prof.~Christoph Hanck}
  \preauthor{\centering\large\emph}
  \postauthor{\par}
  \date{}
  \predate{}\postdate{}


%% linespread settings

\usepackage{setspace}

\onehalfspacing


% Language Setup

\usepackage{ifthen}
\usepackage{iflang}
\usepackage[super]{nth}
\usepackage[ngerman, english]{babel}

%Acronyms
\usepackage[printonlyused, withpage, nohyperlinks]{acronym}
\usepackage{changepage}

% Multicols for the Title page
\usepackage{multicol}


% foot


\begin{document}

\selectlanguage{english}

%%%%%%%%%%%%%% Jens %%%%%
\numberwithin{equation}{section}




\restoregeometry


%%% Header 

\begin{minipage}{0.6\textwidth}
Universität Duisburg-Essen\\
Lehrstuhl Ökonometrie\\
Prof.~Christoph Hanck \\
M.Sc. Jens Klenke \\
\end{minipage}

%\begin{minipage}{0.3\textwidth}
	\begin{flushright}
	\vspace{-3.55cm}
	\includegraphics*[width=5cm]{../Includes/duelogo_de.png}\\
	\vspace{.125cm}
	Wintersemester 2022/2023
	\end{flushright}
%\end{minipage}
%\vspace{.125cm}


\begin{center}
	\vspace{.25cm}
	\textbf{\Large{Deskriptive Statistik}}\\
	\vspace{.25cm}
	\textbf{\large{Übung 6}}\\
	\vspace{.125cm}
\end{center}




% body from markdown

\hypertarget{tansformationseigenschafte-varainz}{%
\section{Tansformationseigenschafte
Varainz}\label{tansformationseigenschafte-varainz}}

Zeigen Sie, dass für \(y_j=\alpha+\beta x_j\), \(j=1,\ldots,n\) gilt

\begin{equation*}
  s_y^2 = \beta^2 s_x^2.
\end{equation*}

\hypertarget{verschiebungssatz}{%
\section{Verschiebungssatz}\label{verschiebungssatz}}

Zeigen Sie den speziellen Verschiebungssatz:
\[s^2 =\frac{1}{n}\sum_{j=1}^nx_j^2-\bar{x}^2\]

\hypertarget{kurtosis}{%
\section{Kurtosis}\label{kurtosis}}

Betrachte erneut die Daten zur Unternehmenszugehörigkeit2:
\begin{equation*}
x_j = \{9, 7, 7, 9, 8, 6, 6, 5, 8, 6, 5, 6, 7, 5, 6, 5, 5, 5, 5, 5\}.
\end{equation*}

Berechnen Sie den zentrierten Kurtosisparameter.

\newpage

\hypertarget{kurtosis-bei-huxe4ufigkeitsverteilten-daten}{%
\section{Kurtosis bei häufigkeitsverteilten
Daten}\label{kurtosis-bei-huxe4ufigkeitsverteilten-daten}}

Betrachten Sie die folgenden häufigkeitsverteilten Daten und berechnen
Sie den zentrierten Kurtosisparameter.

\begin{center}
  \begin{tabular}{ccrr}
    $i$ &   $x_i$ & $h_i$ & $H(x)$ \\[2mm]
  \toprule
    $1$ & $5$ & $0,30$ & $0,30$ \\[2mm]
    $2$ & $6$ & $0,25$ & $0,55$ \\[2mm]
    $3$ & $7$ & $0,15$ & $0,70$ \\[2mm]
    $4$ & $8$ & $0,20$ & $0,90$ \\[2mm]
    $5$ & $9$ & $0,10$ & $1$ \\[2mm]
  \end{tabular}
\end{center}

\newpage

Berechnen Sie für diesen Datensatz:

\begin{enumerate}[label=(\alph*)]
  \item Spannweite 
  \item Quartilsabstand 
  \item Varianz
\end{enumerate}

\end{document}
