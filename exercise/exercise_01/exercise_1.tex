\documentclass[12pt,a4paper]{article}
\usepackage{lmodern}

\usepackage{scalerel}
\usepackage{bbm}
\usepackage{enumitem}
\usepackage{placeins}
\usepackage{amssymb,amsmath}
\usepackage{ifxetex,ifluatex}
\usepackage{fixltx2e} % provides \textsubscript
\ifnum 0\ifxetex 1\fi\ifluatex 1\fi=0 % if pdftex
  \usepackage[T1]{fontenc}
  \usepackage[utf8]{inputenc}
\else % if luatex or xelatex
  \ifxetex
    \usepackage{mathspec}
    \usepackage{xltxtra,xunicode}
  \else
    \usepackage{fontspec}
  \fi
  \defaultfontfeatures{Mapping=tex-text,Scale=MatchLowercase}
  \newcommand{\euro}{€}
\fi
% use upquote if available, for straight quotes in verbatim environments
\IfFileExists{upquote.sty}{\usepackage{upquote}}{}
% use microtype if available
\IfFileExists{microtype.sty}{%
\usepackage{microtype}
\UseMicrotypeSet[protrusion]{basicmath} % disable protrusion for tt fonts
}{}
\usepackage[lmargin = 2cm, rmargin = 2.5cm, tmargin = 2cm, bmargin =
2.5cm]{geometry}


% Figure Placement:
\usepackage{float}
\let\origfigure\figure
\let\endorigfigure\endfigure
\renewenvironment{figure}[1][2] {
    \expandafter\origfigure\expandafter[H]
} {
    \endorigfigure
}

%%%% Jens %%%%
\usepackage{titlesec}
\DeclareMathOperator*{\argmax}{arg\,max}
\DeclareMathOperator*{\argmin}{arg\,min}
\renewcommand{\vec}{\operatorname{vec}}
\newcommand{\tr}{\operatorname{tr}}
\newcommand{\Var}{\operatorname{Var}} % Variance
\newcommand{\MSE}{\operatorname{MSE}} % Variance
\newcommand{\VAR}{\operatorname{VAR}} % Vector autoregression
\newcommand{\Lag}{\operatorname{L}} % Lag operator
\newcommand{\Cov}{\operatorname{Cov}}
\newcommand{\diag}{\operatorname{diag}}
\newcommand{\adj}{\operatorname{adj}}
\newcommand{\loglik}{\operatorname{ll}}


\usepackage{booktabs}
\usepackage{centernot}
\usepackage{mathtools}

\allowdisplaybreaks

\titleformat{\section}
{\normalfont\large\bfseries}{\thesection}{1em}{}

\newcommand{\tmpsection}[1]{}
\let\tmpsection=\section
\renewcommand{\section}[1]{\tmpsection{\underline{#1}} }

\providecommand{\tightlist}{%
  \setlength{\itemsep}{0pt}\setlength{\parskip}{0pt}}

%% citation setup
\usepackage{csquotes}

\usepackage{graphicx}
\makeatletter
\def\maxwidth{\ifdim\Gin@nat@width>\linewidth\linewidth\else\Gin@nat@width\fi}
\def\maxheight{\ifdim\Gin@nat@height>\textheight\textheight\else\Gin@nat@height\fi}
\makeatother
% Scale images if necessary, so that they will not overflow the page
% margins by default, and it is still possible to overwrite the defaults
% using explicit options in \includegraphics[width, height, ...]{}
\setkeys{Gin}{width=\maxwidth,height=\maxheight,keepaspectratio}
\ifxetex
  \usepackage[setpagesize=false, % page size defined by xetex
              unicode=false, % unicode breaks when used with xetex
              xetex]{hyperref}
\else
  \usepackage[unicode=true, linktocpage = TRUE]{hyperref}
\fi
\hypersetup{breaklinks=true,
            bookmarks=true,
            pdfauthor={Prof.~Christoph Hanck},
            pdftitle={Deskriptive Statistik},
            colorlinks=true,
            citecolor=black,
            urlcolor=black,
            linkcolor=black,
            pdfborder={0 0 0}}
\urlstyle{same}  % don't use monospace font for urls
\setlength{\parindent}{0pt}
\setlength{\parskip}{6pt plus 2pt minus 1pt}
\setlength{\emergencystretch}{3em}  % prevent overfull lines
\setcounter{secnumdepth}{5}

%%% Use protect on footnotes to avoid problems with footnotes in titles
\let\rmarkdownfootnote\footnote%
\def\footnote{\protect\rmarkdownfootnote}

%%% Change title format to be more compact
\usepackage{titling}

% Create subtitle command for use in maketitle
\newcommand{\subtitle}[1]{
  \posttitle{
    \begin{center}\large#1\end{center}
    }
}

\setlength{\droptitle}{-2em}
  \title{Deskriptive Statistik}
  \pretitle{\vspace{\droptitle}\centering\huge}
  \posttitle{\par}
\subtitle{Übung 1}
  \author{Prof.~Christoph Hanck}
  \preauthor{\centering\large\emph}
  \postauthor{\par}
  \date{}
  \predate{}\postdate{}


%% linespread settings

\usepackage{setspace}

\onehalfspacing


% Language Setup

\usepackage{ifthen}
\usepackage{iflang}
\usepackage[super]{nth}
\usepackage[ngerman, english]{babel}

%Acronyms
\usepackage[printonlyused, withpage, nohyperlinks]{acronym}
\usepackage{changepage}

% Multicols for the Title page
\usepackage{multicol}


% foot


\begin{document}

\selectlanguage{english}

%%%%%%%%%%%%%% Jens %%%%%
\numberwithin{equation}{section}




\restoregeometry


%%% Header 

\begin{minipage}{0.6\textwidth}
Universität Duisburg-Essen\\
Lehrstuhl Ökonometrie\\
Prof.~Christoph Hanck \\
M.Sc. Jens Klenke \\
\end{minipage}

%\begin{minipage}{0.3\textwidth}
	\begin{flushright}
	\vspace{-3.55cm}
	\includegraphics*[width=5cm]{../Includes/duelogo_de.png}\\
	\vspace{.125cm}
	Wintersemester 2022/2023
	\end{flushright}
%\end{minipage}
%\vspace{.125cm}


\begin{center}
	\vspace{.25cm}
	\textbf{\Large{Deskriptive Statistik}}\\
	\vspace{.25cm}
	\textbf{\large{Übung 1}}\\
	\vspace{.125cm}
\end{center}




% body from markdown

\hypertarget{summen-und-produkte}{%
\section{Summen und Produkte}\label{summen-und-produkte}}

In dieser Aufgabe soll der Umgang mit Produkten und Summen, die in der
Statistik sehr häufig verwendet werden, in Erinnerung gerufen werden.

\begin{enumerate}[label=(\alph*)]
  \item Gegeben sind:
    \begin{center}
      \begin{tabular}{l c c c c}
        \toprule
          $i$   & 1 & 2 & 3 & 4 \\ 
          \midrule
          $x_i$ & 6 & 4 & 1 & 3 \\
          $y_i$ & 1 & 3 & 4 & 2 \\
        \bottomrule
      \end{tabular}
    \end{center}


Berechnen Sie: $$\sum_{i=1}^4 x_i, \qquad \sum_{i=1}^4 x_iy_i, \qquad \prod_{i=1}^4 x_i, \qquad \prod_{i=1}^4 x_iy_i, \qquad \prod_{i=1}^4 x_i^2y_i^{0.5}$$.



\item Berechnen Sie möglichst einfach
  \begin{enumerate}[label=(\roman*)]
  \setlength\itemsep{10px}
    \item $\displaystyle \sum_{i=1}^{20} (6 - 4i) + \sum_{i=1}^{20} (2i + 2)$

    \item $\displaystyle \sum_{i=1}^{30} (i^2+2i-3) + \sum_{i=1}^{30} (3i^2+5i + 8)$

    \item $\displaystyle \sum_{i=1}^{40} (1+i)^2 + \sum_{i=1}^{40} (1-i)^2$
  \end{enumerate}
\vspace{0.5cm}

Beachten Sie dabei, dass gilt:
\begin{eqnarray*}
1+2+3+\ldots+n &=& \sum_{i=1}^n i = \frac{n(n+1)}{2}\\
1^2+2^2+3^2+\ldots+n^2 &=& \sum_{i=1}^n i^2 = \frac{n(n+1)(2n+1)}{6}\\
\end{eqnarray*}

\item Gegeben ist die folgende Matrix $B=(b_{ij})$; $i=1,\ldots,I$ ist der Zeilenindex und $j=1,\ldots,J$ ist der Spaltenindex: \\
\begin{center}
$B= \left(
  \begin{array}{cccccc}
1 & 4 & 4 & 7 & 8 & 4 \\
2 & 3 & 6 & 6 & 2 & 3 \\
6 & 9 & 7 & 6 & 7 & 2 \\
5 & 7 & 8 & 8 & 9 & 6 \\
4 & 6 & 2 & 3 & 4 & 5 \\
3 & 5 & 2 & 3 & 7 & 7 \\
\end{array}
\right)$
\end{center}
\end{enumerate}

Berechnen Sie:

\[\sum_{i=1}^2\sum_{j=1}^3 b_{ij}, \quad \sum_{i=1}^2\sum_{j=1}^J b_{ij}, \quad \sum_{j=1}^J b_{2j}, \quad \sum_{i=1}^I\sum_{j=1}^2 b_{ij}, \quad \sum_{i=3}^4\sum_{j=5}^6 b_{ij}\].

\vspace{.5cm}

\hypertarget{merkmale}{%
\section{Merkmale}\label{merkmale}}

\begin{enumerate}[label=(\alph*)]
  \item Die Studierenden einer Universität verteilen sich wie folgt auf fünf Fakultäten:
  
  \begin{center}
    \begin{tabular}{lcccccr}
      \textbf{Fakultät}    & A   & B  & C   & D   & E   & $\sum\quad$ \\
        \midrule
      \textbf{Studierende} & 100 & 50 & 450 & 200 & 200 & $\quad$\textbf{1.000} \\
    \end{tabular}
  \end{center}
  
  

Geben Sie für die beiden Merkmale $M_1$: \textit{Fakultät} und $M_2$: \textit{Anzahl der Studierenden pro Fakultät} jeweils die Merkmalsausprägungen, die Merkmalsträger und die Klassifikation des Merkmals an.\\
\item Geben Sie bei den nachfolgenden Variablen an, welches Skalenniveau sie besitzen:\\
Geschlecht, Beruf, Warengruppe, Universitätsnoten, Einkommen, Vermögen.
\end{enumerate}

\end{document}
