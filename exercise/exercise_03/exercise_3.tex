\documentclass[12pt,a4paper]{article}
\usepackage{lmodern}

\usepackage{scalerel}
\usepackage{bbm}
\usepackage{enumitem}
\usepackage{placeins}
\usepackage{amssymb,amsmath}
\usepackage{ifxetex,ifluatex}
\usepackage{fixltx2e} % provides \textsubscript
\ifnum 0\ifxetex 1\fi\ifluatex 1\fi=0 % if pdftex
  \usepackage[T1]{fontenc}
  \usepackage[utf8]{inputenc}
\else % if luatex or xelatex
  \ifxetex
    \usepackage{mathspec}
    \usepackage{xltxtra,xunicode}
  \else
    \usepackage{fontspec}
  \fi
  \defaultfontfeatures{Mapping=tex-text,Scale=MatchLowercase}
  \newcommand{\euro}{€}
\fi
% use upquote if available, for straight quotes in verbatim environments
\IfFileExists{upquote.sty}{\usepackage{upquote}}{}
% use microtype if available
\IfFileExists{microtype.sty}{%
\usepackage{microtype}
\UseMicrotypeSet[protrusion]{basicmath} % disable protrusion for tt fonts
}{}
\usepackage[lmargin = 2cm, rmargin = 2.5cm, tmargin = 2cm, bmargin =
2.5cm]{geometry}


% Figure Placement:
\usepackage{float}
\let\origfigure\figure
\let\endorigfigure\endfigure
\renewenvironment{figure}[1][2] {
    \expandafter\origfigure\expandafter[H]
} {
    \endorigfigure
}

%%%% Jens %%%%
\usepackage{titlesec}
\DeclareMathOperator*{\argmax}{arg\,max}
\DeclareMathOperator*{\argmin}{arg\,min}
\renewcommand{\vec}{\operatorname{vec}}
\newcommand{\tr}{\operatorname{tr}}
\newcommand{\Var}{\operatorname{Var}} % Variance
\newcommand{\MSE}{\operatorname{MSE}} % Variance
\newcommand{\VAR}{\operatorname{VAR}} % Vector autoregression
\newcommand{\Lag}{\operatorname{L}} % Lag operator
\newcommand{\Cov}{\operatorname{Cov}}
\newcommand{\diag}{\operatorname{diag}}
\newcommand{\adj}{\operatorname{adj}}
\newcommand{\loglik}{\operatorname{ll}}


\usepackage{booktabs}
\usepackage{centernot}
\usepackage{mathtools}

\allowdisplaybreaks

\titleformat{\section}
{\normalfont\large\bfseries}{\thesection}{1em}{}

\newcommand{\tmpsection}[1]{}
\let\tmpsection=\section
\renewcommand{\section}[1]{\tmpsection{\underline{#1}} }

\providecommand{\tightlist}{%
  \setlength{\itemsep}{0pt}\setlength{\parskip}{0pt}}

%% citation setup
\usepackage{csquotes}

\usepackage{graphicx}
\makeatletter
\def\maxwidth{\ifdim\Gin@nat@width>\linewidth\linewidth\else\Gin@nat@width\fi}
\def\maxheight{\ifdim\Gin@nat@height>\textheight\textheight\else\Gin@nat@height\fi}
\makeatother
% Scale images if necessary, so that they will not overflow the page
% margins by default, and it is still possible to overwrite the defaults
% using explicit options in \includegraphics[width, height, ...]{}
\setkeys{Gin}{width=\maxwidth,height=\maxheight,keepaspectratio}
\ifxetex
  \usepackage[setpagesize=false, % page size defined by xetex
              unicode=false, % unicode breaks when used with xetex
              xetex]{hyperref}
\else
  \usepackage[unicode=true, linktocpage = TRUE]{hyperref}
\fi
\hypersetup{breaklinks=true,
            bookmarks=true,
            pdfauthor={Prof.~Christoph Hanck},
            pdftitle={Deskriptive Statistik},
            colorlinks=true,
            citecolor=black,
            urlcolor=black,
            linkcolor=black,
            pdfborder={0 0 0}}
\urlstyle{same}  % don't use monospace font for urls
\setlength{\parindent}{0pt}
\setlength{\parskip}{6pt plus 2pt minus 1pt}
\setlength{\emergencystretch}{3em}  % prevent overfull lines
\setcounter{secnumdepth}{5}

%%% Use protect on footnotes to avoid problems with footnotes in titles
\let\rmarkdownfootnote\footnote%
\def\footnote{\protect\rmarkdownfootnote}

%%% Change title format to be more compact
\usepackage{titling}

% Create subtitle command for use in maketitle
\newcommand{\subtitle}[1]{
  \posttitle{
    \begin{center}\large#1\end{center}
    }
}

\setlength{\droptitle}{-2em}
  \title{Deskriptive Statistik}
  \pretitle{\vspace{\droptitle}\centering\huge}
  \posttitle{\par}
\subtitle{Übung 3}
  \author{Prof.~Christoph Hanck}
  \preauthor{\centering\large\emph}
  \postauthor{\par}
  \date{}
  \predate{}\postdate{}


%% linespread settings

\usepackage{setspace}

\onehalfspacing


% Language Setup

\usepackage{ifthen}
\usepackage{iflang}
\usepackage[super]{nth}
\usepackage[ngerman, english]{babel}

%Acronyms
\usepackage[printonlyused, withpage, nohyperlinks]{acronym}
\usepackage{changepage}

% Multicols for the Title page
\usepackage{multicol}


% foot


\begin{document}

\selectlanguage{english}

%%%%%%%%%%%%%% Jens %%%%%
\numberwithin{equation}{section}




\restoregeometry


%%% Header 

\begin{minipage}{0.6\textwidth}
Universität Duisburg-Essen\\
Lehrstuhl Ökonometrie\\
Prof.~Christoph Hanck \\
M.Sc. Jens Klenke \\
\end{minipage}

%\begin{minipage}{0.3\textwidth}
	\begin{flushright}
	\vspace{-3.55cm}
	\includegraphics*[width=5cm]{../Includes/duelogo_de.png}\\
	\vspace{.125cm}
	Wintersemester 2022/2023
	\end{flushright}
%\end{minipage}
%\vspace{.125cm}


\begin{center}
	\vspace{.25cm}
	\textbf{\Large{Deskriptive Statistik}}\\
	\vspace{.25cm}
	\textbf{\large{Übung 3}}\\
	\vspace{.125cm}
\end{center}




% body from markdown

\hypertarget{huxe4ufigkeitsverteilungen-und-quantile}{%
\section{Häufigkeitsverteilungen und
Quantile}\label{huxe4ufigkeitsverteilungen-und-quantile}}

Betrachten Sie erneut die Angaben aus Aufgabe \(1\) des zweiten
Arbeitsblattes zur Dauer der Unternehmenszugehörigkeit (in Jahren) der
Beschäftigten eines Unternehmens.

\[X = \left\{ \ 9, \; 7, \; 7, \; 9, \; 8, \; 6, \; 6, \; 5, \; 8, \; 6, \; 5, \; 6, \; 7, \; 5 , \; 6, \; 5, \; 5, \; 5, \; 5, \; 5 \ \right\}
\]

\begin{enumerate}[label=(\alph*)]
  \item Berechnen Sie für den Beispieldatensatz die Werte der absoluten Häufigkeitssummen- und der empirischen Verteilungsfunktion. 
  \item Stellen Sie diese empirische Verteilungsfunktion grafisch dar. 
  \item Nutzen Sie die Ergebnisse aus (a) und (b), um folgende Fragen zu beantworten:
  \begin{enumerate}[label=\roman*)]
    \item Wie viele Arbeitnehmende sind höchstens $7$ Jahre im Unternehmen angestellt?
    \item Wie hoch ist der Anteil der Arbeitnehmenden, die länger als $5$, aber höchsten $7$ Jahre im Unternehmen angestellt sind?
    \item Wie viele Arbeitnehmende sind weniger als $7$ Jahre im Unternehmen angestellt?
    \item Wie hoch ist der Anteil an Arbeitnehmenden, die mindestens $6$ Jahre im Unternehmen angestellt sind?
    \item Wie viele Arbeitnehmende sind länger als $6$ Jahre im Unternehmen angestellt?
  \end{enumerate}
  \item Berechnen Sie das $0,5$-, das $0,4$- und das $0,3$-Quantil. 
\end{enumerate}

\hypertarget{quantile}{%
\section{Quantile}\label{quantile}}

Auf Moodle finden Sie den Datensatz \texttt{hotel\_stars.RData}, welcher
504 Sternebewertungen von Hotelgästen zu deren Aufenthalte in Hotels in
Las Vegas enthält.

\textbf{Hinweise:} Sie können den Datensatz mit dem Befehl
\texttt{load(\textquotesingle{}hotel\_stars.RData\textquotesingle{})} in
Ihre globale Umgebung laden. Achten Sie darauf, dass Ihr Skript und die
Daten im selben Ordner auf Ihrem Rechner gespeichert sind.

Lösen Sie die folgenden Aufgaben ausschließlich in \texttt{R}.

\begin{enumerate}[label=(\alph*)]
  \item Berechnen Sie für den Datensatz die Werte der absoluten Häufigkeitssummenfunktion und der empirischen Verteilungsfunktion.
  \item Plotten Sie die empirische Verteilungsfunktion. 
  \item Berechnen Sie den Modus.
  \item Berechnen Sie den Median.
  \item Berechnen Sie alle Perzentile. 
\end{enumerate}

\end{document}
